\chapter{Uvod}

Kako tehnika sekvencioniranja neprestano napreduje i biva sve jeftinija, broj sekvenci koje se dodaju u biološke baze raste eksponencijalno.  Za dati skup povezanih bioloških sekvenci prvi i najvažniji korak je njihovo poređenje. Poređenje dve ili više sekvenci se radi uz pomoć procedure poravnanja sekvenci. Ova procedura se odnosi na međusobno istraživanje stepena sličnosti sekvenci, njihovih obrazaca konzervacije i evolucione veze koje međusobno dele.

\subsection{Homologija, sličnost I identitet:}
Kada govorimo o dve sekvence obično nas zanima kakvi odnosi važe između njih. Izrazi homologija i sličnost se često koriste kao sinonimi, ali ova dva termina predstavljaju različite odnose. Homologija je kvalitativni termin koji se koristi da opiše zajedničko evoluciono poreklo pri tome ne navodeći koliki je zaista nivo srodnosti. Homologne sekvence mogu se opisati bilo kao ortologe ili paraloge. Ortologi su one sekvence koje su prisutne u različitim vrstama i imaju isto poreklo dok su paraloge sekvence koje su prisutne u isim vrstama i pojavile su se zbog dupliranja gena. Da bi opisali povezanost kvantitativno koristimo identifikaciju i sličnost. Identifikacija se odnosi na to da li su dve sekvence (ili delovi sekvence) evolucione invarijante, dok sličnost potvrđuje zamenu koja čuva strukturne ili funkcionalne uloge. Sličnost između dve sekvence je rezultat koji odražava kako identitet i supsitucija uključuju slične bazne/amino kiseline.


\subsection{Supstitucija i Homologe sekvence}
Da bi procenili sličnosti ili identičnosti koju neke dve sekvence dele potrebno je da sekvence budu poravnate. Poravnanje dve sekvence se naziva parovi poravnanja i može da se korisit za računanje sličnosti bazirano na pogotcima, promašajima ili prazninama. Mutacije koje se akumuliraju u nizu tokom evolucije su supstitucija, insercija i delecija. Supstitucija je rezultat promene u nukleotidu ili amino kiselini. Insercija i delecija (poznati zajedno kao homologe sekvence) označavaju dodavanje ili uklanjanje ostataka tipično predstavljeno crticom. Niz od jedne ili više homologe sekvence je poznatiji kao praznina u poravnanju i on se obično dosta kaznjava. Najkorišćeniji metod za računanje kazni kod praznina je kazna afine praznine.  Ovaj metod radi tako što u rezultat praznine saberemo cenu pojavljivanja praznina i pomnožimo koliko puta se pojavilo praznina za redom sa cenom pojavljivanja vezanih praznina. Ovo znači ako imamo prazninu dužine n u poravnanju, G cena pojavljivanja praznine i L cena ponavljanja praznine onda je ukupna cena jednaka G + L*n. Tipične vrednosti za pojavljivanje praznine je između 10-15 a ponavljanje 1-2. 

Drugi metod za za računanje kazni kod praznina je linearna kazna praznine. Ovaj metod je sličan kao prethodni samo što nema posebnu cenu za prvo pojavljivanje a posebnu za ponavljanje već ima samo jednu cenu.
 
U parovima poravnanja proteina, neki poravnati ostatci mogu biti slični ali ne identični. Ovi slični ostatci su označeni sa „+“ u poravnanju i nazivaju se konzervativne supstitucije.