\chapter{Uvod}

Kako tehnike sekvencioniranja neprestano napreduju i bivaju sve jeftinije, broj sekvenci koje se dodaju u biološke baze eksponencijalno raste.  Za dati skup povezanih bioloških sekvenci prvi i najvažniji korak je njihovo poređenje. Poređenje dve ili više sekvenci radi se uz pomoć procedure poravnanja sekvenci. Ova procedura se odnosi na međusobno istraživanje stepena sličnosti sekvenci, njihovih obrazaca konzervacije i evolucione veze koje međusobno dele.

\subsection{Homologija, sličnost I identitet:}
Kada govorimo o dve sekvence obično nas zanima tip njihovih odnosa. Izrazi homologija i sličnost se često koriste kao sinonimi, ali ova dva termina predstavljaju različite odnose. Homologija je kvalitativni termin koji se koristi da opiše zajedničko evolutivno poreklo nenavodeći pri tome koliki je zaista stepen srodnosti. Homologne sekvence mogu biti ortologe ili paraloge. Ortologe su one sekvence one koje su prisutne u različitim vrstama i imaju isto poreklo dok su paraloge sekvence koje su prisutne u istoj vrsti i pojavile su se zbog duplikacije gena. Da bi opisali povezanost kvantitativno koristimo identifikaciju i sličnost. Identifikacija se odnosi na to da li su dve sekvence (ili delovi sekvence) evoluciono invarijantne, dok sličnost potvrđuje substitucije koje čuvaju strukturne ili funkcionalne uloge. Sličnost između dve sekvence je rezultat koji odražava kako identitet i supsitucija uključuju slične bazne/amino kiseline.


\subsection{Supstitucija i Homologe sekvence}
Da bi procenili sličnosti ili identičnosti koju neke dve sekvence dele potrebno je da sekvence budu poravnate. Poravnanje dve sekvence se naziva parovi poravnanja engl. \textit{pairwise alignments}) i može da se koristi za računanje sličnosti bazirano na pogotcima, promašajima ili prazninama. Mutacije koje se akumuliraju u sekvenci tokom evolucije su supstitucije, insercije i delecije. Supstitucije su rezultat promene u nukleotidu ili amino kiselini. Insercije i delecije (poznati zajedno kao indeli) označavaju dodavanje ili uklanjanje ostataka i tipično se predstavljeno crticom. Niz od jednog ili više susednih indela je poznatiji kao praznina u poravnanju i on se obično dosta kaznjava. Najkorišćeniji metod za računanje kazni kod praznina je kazna afine praznine (engl. \textit{affine gap penalty}).  Ovaj metod radi tako što u rezultat praznine saberemo cenu pojavljivanja praznina i pomnožimo koliko puta se pojavilo praznina za redom sa cenom pojavljivanja vezanih praznina. Ovo znači ako imamo prazninu dužine n u poravnanju, G cena pojavljivanja praznine i L cena ponavljanja praznine onda je ukupna cena jednaka G + L*n. Tipične vrednosti za pojavljivanje praznine je između 10-15 a ponavljanje 1-2. 

Drugi metod za za računanje kazni kod praznina je linearna kazna praznine engl. \textit{linear gap penalty}). Ovaj metod je sličan prethodnom samo što nema posebnu cenu za prvo pojavljivanje a posebnu za ponavljanje već ima samo jednu cenu.
 
U parovima poravnanja proteina, neki poravnati ostatci mogu biti slični ali ne identični. Ovi slični ostaci označeni su sa „+“ u poravnanju i nazivaju se konzervativne supstitucije.